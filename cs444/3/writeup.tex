\documentclass[letterpaper,10pt]{article}

\usepackage{graphicx}                                        
\usepackage{amssymb}                                         
\usepackage{amsmath}                                         
\usepackage{amsthm}                                          

\usepackage{alltt}                                           
\usepackage{float}
\usepackage{color}
\usepackage{url}
\usepackage{hyperref}
\usepackage{listings}

\usepackage{balance}


\usepackage{geometry}
\geometry{textheight=8.5in, textwidth=6in}

%random comment

\newcommand{\cred}[1]{{\color{red}#1}}
\newcommand{\cblue}[1]{{\color{blue}#1}}

\newcommand{\toc}{\tableofcontents}

\title{CS 444 Assignment 3}
\author{Jared Wasinger, Soo-Min Yoo}


\parindent = 0.0 in
\parskip = 0.1 in

\begin{document}

\maketitle
\newpage

\section*{Design for Implementing Algorithms}
	We first copied each of the functions out of the LDD3 documentation for the simple block device and its full implementation provided line by line.\\
    For ecryption, I researched various Linux encryption algorithms, and tried the blowfish algorithm from the Linux CryptoAPI. \\
    We initialized the cipher, and implemented encrypt and decrypt functions for each byte written to the device.\\
    
\section*{Version Control Log}
\begin{tabular}{l l l}\textbf{Detail} & \textbf{Author} & \textbf{Description}\\\hline
b02224d & Soo-Min Yoo & Set encryption key\\\hline
5e7e01b & Soo-Min Yoo & minor variable name fix\\\hline
c2864b5 & Soo-Min Yoo & Add global variables and initialization for encryption\\\hline
e8903ef & Soo-Min Yoo & Replace sbull.c with sbd.c, sbull.c doesn't compile\\\hline
d80b826 & Soo-Min Yoo & Add timer that invalidates device, add refs and more includes\\\hline
148a8f3 & Soo-Min Yoo & minor indent fix\\\hline
ea6c36b & Soo-Min Yoo & minor print statement fix\\\hline
544a249 & Soo-Min Yoo & Change sbd.c to sbull.c, add more functions for bio stuff and init/exit\\\hline
3d7a4a1 & Soo-Min Yoo & Add sbd.c for block device driver without encryption\\\hline
d8667d8 & j-wasinger@hotmail.com & first hw3 commit\\\hline

\end{tabular}

\section*{Assignment Questions}

\begin{enumerate}
  \item{What do you think the main point of this assignment is?}\\
	The point of this assignment was for us to learn how to use a poorly documented API to set up and encrypt a block device.
  
  \item{How did you personally approach the problem? Design decisions, algorithm, etc.}\\
	We divided up the problem into two parts - write a RAM disk device driver, and add encryption to block device using the Linux kernel's CryptoAPI.
	We used an updated sbd.c by Pat Patterson as reference for our block device driver. After researching various Linux encryption algorithms, we decided to try the Blowfish algorithm for our encryption code.
  
  \item{How did you ensure your solution was correct? Testing details, for instance.}
	We ensured our solution was correct by trying hardcoding our encryption key, passing keys as module parameters, and using printk() statements to debug our code as usual.
  
  \item{What did you learn?}
  	We learned about how block devices are handled in the kernel, how to derive a solution from a poorly documented Linux CryptoAPI, and how to compile and load up kernel modules inside the VM.
  
\end{enumerate}




\end{document}